\section{Übung 2}

\subsection{Aufgabe 1}

\subsubsection{Verschiebechiffren}

\begin{align}
	E_1: & z \mapsto (z +k_1)~\mod n \\
	E_2: & z \mapsto (z +k_2)~\mod n 
\end{align}

Dann wäre die Verkettung $ E_2 \circ E_1 $ :

\begin{align}
	E_2 \circ E_1 & = E_2(E_1(z)) =  (((z +k_1)~\mod n) +k_2)~\mod n \\
	              & = z+\underbrace{k_1+k_2}_{k_3} ~\mod n \\
	              & = z+k_3 ~\mod n = E_3(z)
\end{align}

Wir folgern daraus, dass eine Verkettung von zwei Verschiebechiffren keine zusätzlichen Gewinn bringt.

\subsubsection{Multiplikative Chiffren}

\begin{align}
	E_1: & z \mapsto (z \cdot  t_1)~\mod n \\
	E_2: & z \mapsto (z \cdot  t_2)~\mod n 
\end{align}

Dann wäre die Verkettung $ E_2 \circ E_1 $ :

\begin{align}
	E_2 \circ E_1 & = E_2(E_1(z)) =  (((z  \cdot t_1)~\mod n)  \cdot t_2)~\mod n \\
	              & = z \cdot \underbrace{t_1 \cdot t_2}_{t_3} ~\mod n \\
	              & = z \cdot t_3 ~\mod n = E_3(z)
\end{align}

Wir folgern daraus, dass eine Verkettung von zwei Multiplikativen Chiffren keine zusätzlichen Gewinn bringt.


\subsubsection{Tauschchiffren}

\begin{align}
	E_1: & z \mapsto (z \cdot t_1 + k_1)~\mod n \\
	E_2: & z \mapsto (z \cdot  t_2 + k_2)~\mod n 
\end{align}

Dann wäre die Verkettung $ E_2 \circ E_1 $ :

\begin{align}
	E_2 \circ E_1 & = E_2(E_1(z)) =  (((z  \cdot t_1 + k_1)~\mod n)  \cdot t_2 +k_2)~\mod n \\
	              & = z \cdot \underbrace{t_1 \cdot t_2}_{t_3} + \underbrace{k_1 \cdot t_2 + k_2}_{k_3} ~\mod n \\
	              & = z \cdot t_3 + k_3~\mod n = E_3(z)
\end{align}

Wir folgern daraus, dass eine Verkettung von zwei Tauschchiffren keine zusätzlichen Gewinn bringt.

\subsection{Aufgabe 2}

\subsubsection*{Berechnen Sie die multiplikativen Inverse zu $3,~5 ~\text{und} ~22~$ in $Z_23$.}

\begin{align*}
	23 & = 7 \cdot 3 + 2 \\
     3 & = 1 \cdot 2 + 1 \\
     2 & = 2 \cdot 1 + 0 
\end{align*}

\begin{align*}
	 1 & = 3 - 2 \\
     1 & = 3 - (23 - 7 \cdot 3 ) = \\
     1 & = \underline{8} \cdot - 1 \cdot 23    
\end{align*}


\begin{align*}
	23 & = 1 \cdot 15 + 8 \\
    15 & = 1 \cdot 8 + 7 \\
     8 & = 1 \cdot 7 + 1 \\     
     7 & = 7 \cdot 1 + 0     
\end{align*}

\begin{align*}
	 1 & = 8 - 7 \\
	 1 & = (23-15) - (15 - 8)        \\
	 1 & = (23-15) - (15 - (23-15))  \\
	 1 & = 23 - 15  - 15 + 23 - 15   \\
	 1 & = \underbrace{\underline{- 3}}_{20} \cdot 15 + 2 \cdot 23 	 	 
\end{align*}

\begin{align*}
	23 & = 1 \cdot 22 + 1 \\
    22 & = 22 \cdot 1 + 0 \\
\end{align*}

\begin{align*}
	 1 & = 1 \cdot 23 \underbrace{ \underline{- 1}}_{22} \cdot 22 \\
\end{align*}

\subsubsection*{Berechnen Sie die multiplikativen Inversen zu $3,~ 15 ~\text{und} ~22 ~\text{in}~ Z_{24}.$}

\begin{align*}
	24 & = 3 \cdot 8 + 0 \\
	&\Rightarrow \neg \exists ~ \text{multiplikatives Inverses}
\end{align*}

\begin{align*}
	24 & = 1 \cdot 15 + 9 \\
 	15 & = 1 \cdot 9 + 6 \\
	9 & = 1 \cdot 6 + 3 \\
	6 & = 2 \cdot 3 + 0 \\
	&\Rightarrow \neg \exists ~ \text{multiplikatives Inverses}
\end{align*}

\begin{align*}
	24 & = 1 \cdot 22 + 2 \\
 	22 & = 11 \cdot 2 + 0 \\
	&\Rightarrow \neg \exists ~ \text{multiplikatives Inverses}
\end{align*}

\subsubsection*{Zeigen Sie, dass $~(n-1)~$ in $Z_n$ bzgl. der Multiplikation zu sich selbst invers ist.}

\begin{align}
	(n-1) \cdot (n-1) & \equiv_n n^2 -2n + 1 \\
					  & \equiv_n 1 \mod ~ n
\end{align}



\subsection{Aufgabe 3}

Alphabet: \verb+ABCDEFGHIJKLMNOPQRSTUVWXYZ+
Alphabet: \verb+UEBRDNWOLKMSIFHTACGJPQVXYZ+

\begin{verse}
Wind Nord-Ost, Startbahn null-drei,\\
Bis hier hör ich die Motoren.\\
Wie ein Pfeil zieht sie vorbei,\\
Und es dröhnt in meinen Ohren.\\
Und der nasse Asphalt bebt,\\
Wie ein Schleier staubt der Regen,\\
Bis sie abhebt und sie schwebt\\
Der Sonne entgegen.\\
Über den Wolken\\
Muß die Freiheit wohl grenzenlos sein.\\
Alle Ängste, alle Sorgen, sagt man,\\
Blieben darunter verborgen und dann\\
Würde, was hier gross und wichtig erscheint,\\
Plötzlich nichtig und klein.\\
\end{verse}




\subsection{Aufgabe 4}

\subsubsection{}
Die Playfair-Verschlüsselung stellt eine Substitution für Buchstaben-Paare dar. 
Es handelt sich um eine bigraphische monoalphabetische Methode.
 Ähnlich wie bei der einfachen (monographischen) Buchstabensubstitution,
  beruhen Methoden zur Entzifferung von Playfair im Wesentlichen auf einer 
  Analyse der Häufigkeitsverteilung hier der Buchstabenpaare (Bigramme).
   In der deutschen Sprache beispielsweise sind die Bigramme "er", "en" und "ch" sehr häufig. Im Beispieltext fallen die "Doppler" (also Bigramm-Wiederholungen) ME…ME, IK…IK, QC…QC und TE…TE sowie die "Reversen" (Wiederholung eines umgedrehten Bigramms) CQ…QC auf, die sich in gleicher Weise im englischen Klartext wiederfinden.
Da kein Buchstabe mit sich selbst gepaart wird, gibt es nur 600 (25×24) mögliche Buchstabenkombinationen, 
die substituiert werden. Überdies gibt es eine Reihe von Symmetrien, die teilweise schon am obigen Beispieltext erkannt werden können.
So hilft der erwähnte Klartext-Geheimtext-Zusammenhang EL ↔ CQ und LE ↔ QC 
beim Bruch des Textes. Ist nämlich ein Bigramm geknackt, dann ist auch sofort das reverse (umgedrehte) Bigramm bekannt.
In den Fällen des Überkreuz-Schrittes gibt es darüber hinaus noch weitere Beziehungen zwischen den vier auftretenden Buchstaben
 in der Art (vgl. beispielsweise obere linke Ecke des Quadrats) DC ↔ EB, CD ↔ BE, EB ↔ DC sowie BE ↔ CD, die der Angreifer zur
  Entzifferung ausnutzen kann. Ferner hat auch die geschilderte Methode zur Erzeugung des Playfair-Quadrats Schwächen, denn
   es endet häufig – wie auch im Beispiel – auf "XYZ".
Die Playfair-Verschlüsselung ist somit weit entfernt von einer allgemeinen bigraphischen Methode
 mit völlig willkürlicher Zuordnung der Buchstabenpaare und stellt in der heutigen Zeit
  kein sicheres Verschlüsselungsverfahren mehr dar. So lassen sich mit modernen Mitteln
   auch relativ kurze Playfair-Texte in sehr kurzer Zeit brechen.


\subsection{Aufgabe 5}

Herleitung des Gleichungssystems: 

\begin{align}
H x_1+Ix_2 &= \ddot{A}  & Lx_1+Lx_2 = U \\
H x_3+Ix_4 &= U  & Lx_3+Lx_4 = K \\
\end{align}

\begin{math}
\begin{pmatrix}
7 & 8 & 0 & 0 \\ 
0 & 0 & 7 & 8 \\ 
11 & 11 & 0 & 0 \\ 
0 & 0 & 11 & 11
\end{pmatrix}
\begin{pmatrix} x_1 \\x_2 \\x_3 \\x_4 \end{pmatrix}
=
\begin{pmatrix} 26 \\ 20 \\ 20 \\ 10 \end{pmatrix}
\end{math}


\begin{align}
\begin{gmatrix}[p]
7 & 8 & 0 & 0   & 26 \\ 
0 & 0 & 7 & 8   & 20 \\ 
11 & 11 & 0 & 0 & 20 \\ 
0 & 0 & 11 & 11 & 10
\rowops
\swap{1}{2}
\end{gmatrix} &
\begin{gmatrix}[p]
7 & 8 & 0 & 0   & 26 \\ 
11 & 11 & 0 & 0 & 20 \\ 
0 & 0 & 7 & 8   & 20 \\ 
0 & 0 & 11 & 11 & 10
\rowops
\mult{0}{\cdot 7^{-1}  = 25 }
\mult{1}{\cdot 11^{-1} =  8 }
\mult{2}{\cdot 7^{-1}  = 25 }
\mult{3}{\cdot 7^{-1}  =  8 }
\end{gmatrix}\\[5mm]
\begin{gmatrix}[p]
1 & 26 & 0 & 0   & 12 \\ 
1 & 1 & 0 & 0 & 15 \\ 
0 & 0 & 1 & 26   & 7 \\ 
0 & 0 & 1 & 1 & 22
\rowops
\add[\cdot -1]{0}{1}
\add[\cdot -1]{2}{3}
\end{gmatrix}&
\begin{gmatrix}[p]
1 & 26 & 0 & 0   & 12 \\ 
0 & 4 & 0 & 0 & 3 \\ 
0 & 0 & 1 & 26   & 7 \\ 
0 & 0 & 0 & 4 & 22
\rowops
\mult{1}{\cdot 4^{-1}=22}
\mult{3}{\cdot 4^{-1}=22}
\end{gmatrix}\\[5mm]
\begin{gmatrix}[p]
1 & 26 & 0 & 0   & 12 \\ 
0 & 1 & 0 & 0 & 8 \\ 
0 & 0 & 1 & 26   & 7 \\ 
0 & 0 & 0 & 1 & 11
\rowops
\add[\cdot -26]{1}{0}
\add[\cdot -26]{3}{2}
\end{gmatrix}&
\begin{pmatrix} 7 \\  8 \\  11 \\  11\end{pmatrix}  =
\begin{pmatrix} H \\  I \\  L \\  L\end{pmatrix} 
\end{align}

\paragraph*{Bildung der Inversen $K^{-1}$}

\begin{align*}
\begin{gmatrix}[p]
 7 &  8 & 1 &0  \\
11 & 11 & 0 &1
\rowops 
\mult{0}{\cdot 7^{-1} = 25}
\end{gmatrix} &
\begin{gmatrix}[p]
 1 &  26 & 25 &0  \\
11 & 11 & 0 &1
\rowops 
\add[\cdot 11^{1}]{0}{1}
\end{gmatrix} \\
\begin{gmatrix}[p]
 1 &  26 & 25 &0  \\
 0 &  15 & 15 &1   
\rowops 
\mult{1}{\cdot 15^{-1} = 2}
\add[\cdot 26]{1}{0}
\end{gmatrix} &
\begin{gmatrix}[p]
1 & 0 & 28 & 6 \\
0 & 1 & 1 & 2
\end{gmatrix} 
\end{align*}

\begin{equation} K^{-1} = 
\begin{pmatrix}
28 & 6 \\ 
1 & 2
\end{pmatrix} 
\end{equation}

\textbf{Lösung: HILLISTEINFACHZUKNACKEN}

\subsection{Aufgabe 6}
\subsubsection{Warum gilt für zufällige Texte $I_r = \dfrac{1}{26} = 0.0385$?}

Wirklicher Zufall würde bedeuten das jeder Buchstaben $a \in A$ gleich oft im Text vorkommt.
Folglich handelt es sich um einen Laplace-Raum (wie beim Würfel) und die Warscheinlichkeit für $P(X = a) = \dfrac{1}{|A|}$.
In unseren Fall ist $ |A|=26 $.

%\begin{align}
%I = \dfrac{\sum_{i=1}{26} m_i (m_i-1)}{m(m-1)} }}
%m_1 = m_2 = m_3 = \cdots =  m_26 \\
%I = 0,0469
%\end{align}

\subsubsection{}
\subsubsection{}

\subsection{Übungsaufgabe: One-Time-Pad}

\begin{align}
P_1 &= hike = 001~ 010~ 011~ 000  \\
P_2 &= rike = 101~ 010~ 011~ 000  \\
C   &= eier = 000~ 010~ 000~ 101  \\
K   &= klet = 011~ 100~ 000~ 111  \\
\end{align}

\subsubsection{}

\begin{align}
 001~ 010~ 011~ 000 &\xor \\
 000~ 010~ 000~ 101 &=   \\
 001~ 000~ 011~ 101 &= hekr
\end{align}

\begin{align}
 101~ 010~ 011~ 000 &\xor \\
 000~ 010~ 000~ 101 &=   \\
 101~ 000~ 011~ 101 &= rekr
\end{align}

\subsubsection{}

\begin{align}
 001~ 010~ 011~ 000 &\xor \\
 011~ 100~ 000~ 111 &=   \\
 010~ 110~ 011~ 111 &= iskt
\end{align}

\begin{align}
 101~ 010~ 011~ 000~ &\xor \\
 011~ 100~ 000~ 111~ &=   \\
 110~ 110~ 011~ 010~ &= sski
\end{align}

\subsubsection{}

wenn $C_{1_i} = C_{2_i} \Rightarrow P_{1_i} = P_{2_i}$ 

\begin{equation}
C_1 \xor C_2 = (P_1 \xor K) \xor (P_2 \xor K) = P_1 \xor P_2 
\end{equation}

\subsubsection{}

\begin{align}
K &= C_1 \xor P_1 \\
C_2 \xor K &= P_2
\end{align}


\subsection{Skytale}
\subsubsection{}

\begin{align}
E(k, x_1 , \ldots , x_{km} ) &= \\
x_1 x_{m+1} x_{2m+1} \ldots x_{(k−1)m+1} x_{2} x_{m+2} x_{2m+2} \ldots x_{(k−1)m+2} \ldots x_m x_{2m} x_{3m} \ldots x_km
\\
\begin{pmatrix}
x_1 & x_2 & \cdots & x_m \\ 
x_{m+1} & x_{m+2} & \cdots & x_{2m} \\ 
\vdots & \vdots & \ddots & \vdots \\ 
x_{(k-1)m+1} & x_{(k-1)m+2} & \cdots & x_{km}
\end{pmatrix} 
\end{align}


%f(x) = K * x mod (N'-1)
%N' = K [N/K] für x \in N x startet bei 0
%f(x) = N/K xmod (N-1)

Ist die Klartextlänge kein Vielfaches von k, so kann der Klartext durch das Ein- bzw.
Anfügen von sogenannten Blendern (Füllzeichen) verlängert werden. Damit der Emp\-
fänger diese Füllzeichen nach der Entschlüsselung wieder entfernen kann, ist lediglich
darauf zu achten, dass sie im Klartext leicht als solche erkennbar sind.
