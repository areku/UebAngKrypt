\section{Kryptologische Anwendungen und Protokolle}

\subsection{Münzwurf am Telefon}
Alice sendet $n = 34189$ an Bob. Bob wählt $x = 17209$ und sendet $4563$ an Alice.
\subsubsection{Alice sendet $r = 16980$ an Bob. Kann Bob n faktorisieren? Wenn ja, geben Sie die
Faktorisierung an.}

\textbf{Ist $r=n-x$:} $n-r=x = 17209$. Faktorisierung nicht möglich, Alice gewinnt

\subsubsection{Alice sendet $23474$ an Bob. Kann Bob n faktorisieren? Wenn ja, geben Sie die
Faktorisierung an.}

$ggT(x+y,n) = ggT(23474+17209, 34189) = 191$

o.\,B.\,d.\,A.:~$q = \dfrac{n}{p} = 179, p = 191$ 

\subsubsection{Berechnen Sie die vier Quadratwurzeln von $17209 \mod n$.}

Es gibt kein $x$, sodass $x^2 \equiv 17209 \mod 34189$


\subsection{Altersvergleich}
\subsection{Karten kryptologisch mischen und austeilen}
Gegeben seien die folgenden vier Spielkarten und Ihre Codierung:
\begin{align}
x_1="Herz Ass" &= 2  &     x_2="Pik Ass"  &= 3 \\
x_3="Karo Ass" &= 4  &     x_4="Kreuz Ass"&= 7
\end{align}
Mischen und verteilen Sie die Karten nach der in der Vorlesung beim Skat-Protokoll
vorgestellten Methode an die vier Spieler $A, B, C, D$.
Verwenden Sie folgende Permutationen in Zykelschreibweise:
% = (124), =(14)(23), =(134), =id (Zykelschreibweise)
Ferner sei $p=11, a=3, b=7, c=9, d=7$.


\subsection{Knobeln über E-Mail}
Entwerfen Sie ein (dezentrales) Protokoll zum Knobeln (Schere, Stein, Papier) per E-
Mail. Hinweis: Es gibt verschiedene Lösungen. Sie können z.B. Shamirs No-Key-Protokoll
oder Hashfunktionen verwenden.

\subsection{Chaffing and Winnowing}
\subsubsection{Welche Tripel (Seriennummer, Paket, MAC) übertragen Sie nach dem
"Chaffing and Winnowing"-Verfahren, wenn die Nachricht "FHT" lautet und die
Paketlänge einen Buchstaben lang ist?}
\subsubsection{Welche Tripel (Seriennummer, Paket, MAC) übertragen Sie nach dem
"Chaffing and Winnowing"-Verfahren, wenn die Nachricht "F" lautet und die
Paketlänge ein Bit lang ist? Verwenden Sie die 8-Bit ASCII-Codierung.}

\textbf{Vorgaben:}
Schlüssel k zur Berechnung des MAC = "geheim"
Hash-Verfahren = MD5
MAC-Verfahren = $H(k, H(k, M))$
Hinweis: Sie können zur Generierung der MACS das CrypTool verwenden
(Einzelverfahren => Hashverfahren => Generieren von MACs)
