\section{Kryptologische Anwendungen und Protokolle – Teil 2}

\subsection{Altersvergleich}
Führen Sie das Protokoll zum Altersvergleich mit folgenden Parametern durch.
$ 0 \le a, b \le 4$, der öffentliche RSA-Schlüssel von Bob lautet $n=143, e=19, x = 102$.
Als Einwegfunktion verwenden Sie eine Reduktion $\mod 53$.
Geben Sie die notwendigen Berechnungen, die ausgetauschten Nachrichten und das
Ergebnis für folgende Alter von Alice und Bob an:

\begin{align}
	p &= 11 & q &= 13 \\
	d &= 19
\end{align}

\subsubsection{$a=1, b=1$}

\begin{tabular}{ccc}
	Alice 	&& Bob                       \\
	$x=102$                             \\
	$c = E_{PK-B}(x)=102^{19} \mod 143$ \\
	$d = c-a = 15 - 1$     & \rarr &       \\
    a & b &   $y &= D_{SK_B}(14+0,14+1,14+2,14+3,14+4)$ \\
	a & b &   $~ &= (92, 102, 42, 134, 8) $\\
	a & b &   $z &= (39, 49, 42, 28, 8)$	  \\
    a &\larr & $(39, 49, 42+1, 28+1, 8+1)$\\
	$f(x)=49 \in z$ & a & a
\end{tabular}

\subsubsection{a=1, b=3}

\subsubsection{a=1, b=0}

Angenommen Bob verzichtet leichtsinnigerweise auf die Anwendung der
Einwegfunktion. Zeigen Sie im Fall b) wie Alice Bobs Alter rekonstruieren kann.


\subsection{No-Key-Protokoll}
Führen Sie das No-Key-Protokoll mit folgenden Parametern durch:
p = 17, a = 3, b = 5 und s = 2
Skizzieren Sie den Protokollablauf, berechnen Sie die ausgetauschten Werte und
rekonstruieren Sie das Geheimnis.

\subsection{(4,6)-Schwellwertverfahren über Gleichungssystem}
Es seien p=19 und die folgenden Wertepaare gegeben: (1,7), (2,3), (3,4), (16, 4),
(17,5) und (18,1). Rekonstruieren Sie aus (1,7), (2,3), (17,5) und (18,1) das
Geheimnis und das Polynom durch Lösen des entsprechenden linearen
Gleichungssystems.
Zeichnen Sie das Polynom für die Wert von x=0 bis x=20.


\subsection{(3,4)-Schwellwertverfahren über Lagrange}
Gegeben sind folgende Punkte (1,6) (2,3), (3,2), (4,3).
Wählen Sie drei Punkte aus und rekonstruieren Sie das Polynom über das
Lagrangesche-Interpolationspolynom.
Was ist das Geheimnis? Zeichen Sie das Polynom.
